Gold et al. thematisierten die Frage, ``zu welchem Zeitpunkt, vor dem Auftreten einer Systemgrenze, muss das FAS die Aufmerksamkeit des Fahrers auf sich ziehen, um eine erfolgreiche Übernahme durch den Fahrer auch dann sicherzustellen, wenn er sich nicht im Loop befindet.'' \cite{gold} %
Hierzu wurde eine Studie mit 32 Probanden in einem High-Fidelity-Fahrsimulator bei BMW durchgeführt. Das Szenario beschreibt eine Fahrt mit 120 km/h auf einer dreispurigen Autobahn, bei der das vorausfahrende Auto einen Unfall verursacht und somit den Fahrer zu handeln zwingt. 17 Probanden haben dieses Szenario als Referenz manuell durchfahren, wobei der Unfall zum Einen fünf Sekunden, zum Anderen 7 Sekunden entfernt war. Die gleichen Zeiten wurden für die Probanden der hochautomatisierten Zeit eingehalten um die Übergabe einzuleiten.  

Folgende Ergebnisse konnten aus dieser Studie gezogen werden:
\begin{itemize}
	\item[1.] Die geringe Zeit bis zur Übernahme lässt die Probanden schneller zu einer Entscheidung und Reaktion kommen, dennoch ist deren Qualität schlecht.
	\item[2.] Mit der sich verringernden Zeit bis zur Übernahme nehmen die Kontrollblicke in Spiegel und über die Schulter ab, hingegen nimmt die Beschleunigung zu, sowie auch das Betätigen der Bremse.
	\item[3.] Vergleicht man die Probanden aus der Referenzfahrt und der hochautomatisierten Fahrt, wird deutlich, dass bei den Probanden der hochautomatisierten Fahrt bis zu dreimal so hohe Beschleunigungen erzielt wurden. Auch werden hier viele plötzliche Bremsmanöver durchgeführt. 
\end{itemize}

Die Studie belegt unter diesen experimentellen Bedingungen, dass bei vollständiger Ablenkung des Fahrers bei hochautomatisierter Fahrt noch bei sieben Sekunden Übernahmezeit ein Automatisierungseffekt auftritt. Das bedeutet, dass der Fahrer durch die fahr-fremde Nebentätigkeit stark abgelenkt ist und der automatisierten Fahrt vertraut. Dies führt zu solchen Reaktionen bei unerwarteter Bekanntmachung von Übernahmen. 






